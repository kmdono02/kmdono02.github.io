%%%%%%%%%%%%%%%%%%%%%%%%%%%%%%%%%%%%%%%%%
% Medium Length Professional CV
% LaTeX Template
% Version 2.0 (8/5/13)
%
% This template has been downloaded from:
% http://www.LaTeXTemplates.com
%
% Original author:
% Rishi Shah 
%
% Important note:
% This template requires the resume.cls file to be in the same directory as the
% .tex file. The resume.cls file provides the resume style used for structuring the
% document.
%
%%%%%%%%%%%%%%%%%%%%%%%%%%%%%%%%%%%%%%%%%

%----------------------------------------------------------------------------------------
%	PACKAGES AND OTHER DOCUMENT CONFIGURATIONS
%----------------------------------------------------------------------------------------

\documentclass{resume} % Use the custom resume.cls style

\usepackage{longtable}
\usepackage{bibentry}
\usepackage[maxbibnames=99, style=nature]{biblatex}
\usepackage{hyperref}

\usepackage[left=0.75in,top=0.6in,right=0.75in,bottom=0.6in]{geometry} % Document margins
\newcommand{\tab}[1]{\hspace{.2667\textwidth}\rlap{#1}}
\newcommand{\itab}[1]{\hspace{0em}\rlap{#1}}
\name{Kevin Donovan} % Your name
\address{2007 Wallace St., Philadelphia, PA 19130} % Your address
\address{(315)727-3603 \\ donovke@pennmedicine.upenn.edu} % Your phone number and email

\addbibresource{CV.bib}
\addbibresource{CV_software.bib}
\addbibresource{CV_talks.bib}

\newcommand*{\boldname}[3]{%
  \def\lastname{#1}%
  \def\firstname{#2}%
  \def\firstinit{#3}}
\boldname{}{}{}

% Patch new definitions
\renewcommand{\mkbibnamegiven}[1]{%
  \ifboolexpr{ ( test {\ifdefequal{\firstname}{\namepartgiven}} or test {\ifdefequal{\firstinit}{\namepartgiven}} ) and test {\ifdefequal{\lastname}{\namepartfamily}} }
  {\mkbibbold{#1}}{#1}%
}

\renewcommand{\mkbibnamefamily}[1]{%
  \ifboolexpr{ ( test {\ifdefequal{\firstname}{\namepartgiven}} or test {\ifdefequal{\firstinit}{\namepartgiven}} ) and test {\ifdefequal{\lastname}{\namepartfamily}} }
  {\mkbibbold{#1}}{#1}%
}

\boldname{Donovan}{Kevin}{K.}

\DeclareFieldFormat[misc]{title}{#1}
\DeclareFieldFormat[proceedings]{title}{#1}

\begin{document}

%----------------------------------------------------------------------------------------
%	Profession Summary
%----------------------------------------------------------------------------------------

\begin{rSection}{Professional Summary}
More than five years of experience in academic settings leading statistical analyses resulting in published research in top-tier scientific journals.  Research included statistical methods development with applications in infectious disease, Alzheimer's, and multiple sclerosis, as well as cross-disciplinary work in Autism Spectrum Disorder and HIV studies.  Experience working in collaborative environments and communicating statistics to members of the broad scientific community.
\end{rSection}

%----------------------------------------------------------------------------------------
%	EDUCATION SECTION
%----------------------------------------------------------------------------------------

\begin{rSection}{Education}

{\bf University of North Carolina at Chapel Hill} \hfill {\em August 2015 - August 2021} 
\\ PhD in Biostatistics
\\ Gillings School of Global Public Health
% \\ Adviser: Dr. Kihn Truong\\

\end{rSection}

%--------------------------------------------------------------------------------
%    Experience
%-----------------------------------------------------------------------------------------------
\begin{rSection}{Experience}
{\bf Postdoctoral Fellow} \hfill {\em September 2021 - Present}
\\ University of Pennsylvania
% \\ Adviser: Dr. Russell T. Shinohara
\begin{itemize}
    \item Development of machine learning algorithm using support vector regression to remove multivariate relationship between a confounder and set of predictor variables.  Applied to medical imaging data in the context of Alzheimer's disease.
    \item Development of analysis pipeline to automatically detect lesion incidence in multiple sclerosis patients using low-resolution medical images.  
    \item Lead statistical analyses on two projects with researchers in Radiology.  First project published with second project under revision.
    \item Instructor for graduate level course serving as an introduction to the application of biostatistical methods.
\end{itemize}
{\bf Research Assistant} \hfill {\em March 2018 - Present}
\\ Carolina Institute for Developmental Disabilities
\begin{itemize}
    \item Development of algorithm for early prediction of Autism Spectrum Disorder (ASD) using behavioral data with random forests using R, published in academic journal.
    \item Analysis focused on examining causes of ASD prevalence and symptom heterogeneity by infant sex, using latent variable models such as factor analysis, growth mixture models and clustering methods.  Resulted in multiple publications.
    \item Direct collaboration with scientists writing statistical analysis and results sections in published manuscripts, and conducting analyses in R.  Methods used include generalized linear models, mixed models with longitudinal data, mediation models, and clustering methods.  Resulted in multiple publications, with analysis code publicly available.
    \item Development of a set of tutorials detailing the use of R for data management and data analysis.  Course based on these tutorials created with bi-weekly virtual sessions held and corresponding office hours.  All made freely available online for public.
    \item Instructor for graduate-level course serving as introduction to machine learning methods, including statistical theory and application to real data.
\end{itemize}
{\bf Research Assistant} \hfill {\em September 2016 - May 2019} 
\\ Collaborative Studies Coordinating Center
\begin{itemize}
    \item Under direction of mentor, lead statistical analyses for published research on HIV-positive youth, collaborating with investigators across the United States.
    \item Development of R package \textbf{lodr} to conduct regression analyses when predictors have a known limit of detection.  Package made publicly available on CRAN.
\end{itemize}
{\bf Research Assistant} \hfill {\em August 2015 - March 2018} 
\\ University of North Carolina at Chapel Hill
\begin{itemize}
    \item Developed and published research on methodology for estimating biomarker levels which correspond to a desired upper bound on the risk of disease.
\end{itemize}

\end{rSection}
%----------------------------------------------------------------------------------------
%	Coursework
%----------------------------------------------------------------------------------------

%\begin{rSection}{Coursework}
%Advanced Probability and Statistical Inference\\
%Linear and Generalized Linear Models\\
%Longitudinal Data Analysis\\
%Statistical Methods in Diagnostic Medicine\\
%Machine Learning\\
%Survival Analysis\\
%Spatial Statistics
%\end{rSection}

%----------------------------------------------------------------------------------------
%	Computing Experience
%----------------------------------------------------------------------------------------

%\begin{rSection}{Computing Experience}
%R, SAS, C++, Rcpp, Matlab, Python, Linux cluster computing, medical image processing software (FSL, ANTS, ITK-Snap)
%\end{rSection}

%----------------------------------------------------------------------------------------
%	Skills
%----------------------------------------------------------------------------------------

\begin{rSection}{Skills}
\textbf{Computing}: R, SAS, C++, Matlab, Python, Linux cluster \\
\textbf{Image processing}: FSL, ANTS, ITK-Snap
\end{rSection}

%----------------------------------------------------------------------------------------
%	Developed Software
%----------------------------------------------------------------------------------------

\begin{rSection}{Developed Software}

\begin{refsection}[CV_software.bib]
\nocite{*}
\leavevmode\printbibliography[omitnumbers=true,heading=none]
\end{refsection}

\end{rSection}

%----------------------------------------------------------------------------------------
%	Publications
%----------------------------------------------------------------------------------------
\begin{rSection}{Publications}

\begin{refsection}[CV.bib]
\nocite{*}
\printbibliography[keyword=Published,omitnumbers=true,title=Published]
\printbibliography[keyword=In Progress,omitnumbers=true,title=In Progress]
\end{refsection}

\end{rSection}

%----------------------------------------------------------------------------------------
%	Professional Service
%----------------------------------------------------------------------------------------

%\begin{rSection}{Professional Service}
%Reviewer for Biometrics, Frontiers in Neuroscience, Biostatistics
%\end{rSection}

%----------------------------------------------------------------------------------------
%	Professional Presentations
%----------------------------------------------------------------------------------------
\begin{rSection}{Professional Presentations}

\begin{refsection}[CV_talks.bib]
\nocite{*}
\leavevmode\printbibliography[omitnumbers=true,heading=none]
\end{refsection}

\end{rSection}

%----------------------------------------------------------------------------------------
%	References
%----------------------------------------------------------------------------------------
\begin{rSection}{References}
Available upon request
\end{rSection}

\end{document}
